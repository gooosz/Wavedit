\section{Frequenzen filtern}
\subsection{Vorgehensweise}
\begin{frame}{\insertsection}
	\framesubtitle{\insertsubsection}
	\begin{enumerate}
		\item Zeitdiskretes Signal durch DFT in Frequenzbereich überführen
		\item Amplitude der zu filternden Frequenzen auf $0$ setzen
		\item IDFT anwenden um das gefilterte zeitdiskrete Signal zu erhalten
	\end{enumerate}
\end{frame}

\subsection{Was kann schon schief gehen?}
\begin{frame}{\insertsection}
	\framesubtitle{\insertsubsection}
	\begin{itemize}
		\vspace{1em}
		\item Multiplizieren der zu filternden Frequenzanteile mit $0$ ist dasselbe wie Rechteck-Fenster anwenden
		\\ $\Rightarrow$ Faltungstheorem greift erneut%\\ IDFT von Rechteck-Fenster ist wieder Sinc-Funktion
		\\ $\Rightarrow$ Frequenzanteile die nicht gefiltert werden sollen, \\ \hskip 1.35em werden auch (leicht) beeinflusst
	\end{itemize}
	\begin{figure}
		\includegraphics[height=0.3\textheight]{images/rectFilter\_impulseresponse.png}
		\caption*{\centering Idealer vs Realisierbarer Bandpass-Filter, \\ \href{https://dsp.stackexchange.com/a/37662}{https://dsp.stackexchange.com/a/37662}}
	\end{figure}
\end{frame}

\begin{frame}{\insertsection}
	\framesubtitle{\insertsubsection}
	\begin{itemize}
		\item Lösung: Fensterfunktion wird auf Rechteck-Fenster angewendet
		\item Frequenzen die nicht gefiltert werden sollen,\\ werden nun kaum beeinflusst
		\item filtert Frequenzen nicht auf $0$,\\ ein kleiner Frequenzanteil bleibt 
	\end{itemize}
\end{frame}